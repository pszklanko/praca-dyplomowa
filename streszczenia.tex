\newpage
\begin{center}
\large \bf
Wykorzystanie protokołu HTTP/2.0 do budowy szybkiej aplikacji internetowej
\end{center}

\section*{Streszczenie}
Praca składa się ze wstępu, w którym informuję o czym jest praca i dlaczego zdecdyowałem się na taki temat.
Opisuję też krótko wybrane technologie oraz biblioteki oraz powody, dla których się na nie zdecydowałem.
Drugi rozdział to krótka historia protokołu HTTP/2 oraz opis elementów, które zostały do niego wprowadzone.
Trzeci rozdział opisuje kluczowe elementy stworzonej aplikacji.
Krok po kroku przedstawia ich implementację.
W rozdziale czwartym przeprowadzam testy porównawcze obu wersji protokołu HTTP.
Dodatkowo sprawdzam też kompatybilność protokołu HTTP/2 z najnowszymi przeglądarkami internetowymi.
W ostatni rodziale podsumowuję wyniki swojej pracy, wyciągam wnioski oraz przedstawiam plany na przyszłość związane z tym projektem.

\bigskip
{\noindent\bf Słowa kluczowe:} protokół, HTTP, HTTP/2, SERVER PUSH

\vskip 2cm


\begin{center}
\large \bf
THESIS TITLE
\end{center}

\section*{Abstract}
%This thesis presents a novel way of using a novel algorithm to solve complex
%problems of filter design. In the first chapter the fundamentals of filter design
%are presented. The second chapter describes an original algorithm invented by the
%authors. Is is based on evolution strategy, but uses an original method of filter
%description similar to artificial neural network. In the third chapter the implementation
%of the algorithm in C programming language is presented. The fifth chapter contains results
%of tests which prove high efficiency and enormous accuracy of the program. Finally some
%posibilities of further development of the invented algoriths are proposed.

\bigskip
{\noindent\bf Keywords:} thesis, LaTeX, quality

\vfill