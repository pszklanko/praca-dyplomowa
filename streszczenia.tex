\newpage
\begin{center}
\large \bf
Wykorzystanie protokołu HTTP/2 do budowy szybkiej aplikacji internetowej
\end{center}

\section*{Streszczenie}
Praca składa się ze wstępu, w którym informuję o czym jest i dlaczego zdecdyowałem się na taki temat.
Opisuję też krótko wybrane technologie, biblioteki oraz powody, dla których się na nie zdecydowałem.
Drugi rozdział to krótka historia protokołu HTTP/2 oraz opis elementów, które zostały do niego wprowadzone.
Skupiam się tutaj na funkcjach protokołu HTTP/2, które nie były znane w poprzedniej wersji, a są kluczowym elementem HTTP/2
Trzeci rozdział opisuje kluczowe elementy stworzonej aplikacji.
Krok po kroku przedstawia ich implementację.
Następnie, na początku rozdziału 4, opisuję narzędzie, z którego korzystałem podczas badania możliwości protokołu HTTP/2.
Po opisaniu narzędzia do testów przedstawiam swoje środowisko testowe.
Jest to aplikacja, którą stworzyłem na potrzeby tego projektu.
W tej części przedstawiam, do czego służy ta aplikacja oraz pokazuję jej główne funkcje.
Przedstawiam również fragmenty kodu, która są odpowiedzialne za wykonywanie kluczowych funkcji w aplikacji.
W drugiej części rozdziału czwartego przeprowadzam testy porównawcze obu wersji protokołu HTTP.
Sprawdzam jak protokół radzi sobie w różnych sytuacjach w porównaniu do HTTP/1.1.
Testuję również funkcję Server Push protokołu HTTP/2
Dodatkowo sprawdzam też kompatybilność protokołu HTTP/2 z najnowszymi przeglądarkami internetowymi.
W ostatni rodziale podsumowuję wyniki swojej pracy, wyciągam wnioski oraz przedstawiam plany na przyszłość związane z tym projektem.

\bigskip
{\noindent\bf Słowa kluczowe:} protokół, HTTP, HTTP/2, Server Push

\vskip 2cm


\begin{center}
\large \bf
Using HTTP/2 protocol for building a fast web application
\end{center}

\section*{Abstract}

The thesis consists of an introduction, in which I describe it and explain why I decided to choose this topic.
Introduction also briefly describes selected technologies, libraries and reasons, why I decided to use them.
The second chapter is a brief history of the HTTP/2 and the description of the elements that have been introduced in it.
I focus here on the features of HTTP/2, which were not known in the previous version, and are a key element of the HTTP/2
The third section describes the key elements of the created application.
Step by step, presents their implementation.
Then, at the beginning of Chapter 4, I describe the tool which I used during the test capabilities of HTTP/2.
After describing testing tools I present my test environment.
It is an application that I created for this project.
In this section, I present source of the application and show its main functions.
Presented code shows functions, which are responsible for performing critical tasks in the application.
In the second part of the fourth chapter I carry out comparative tests of both versions of the HTTP protocol.
I check how the protocol handles in a variety of situations compared to HTTP/1.1.
I also test Push Server -- a new feature of HTTP/2.
Additionally I also check the compatibility of HTTP/2 with the latest web browsers.
In the last section I sum up the results of my work, draw conclusions and present the future plans for this project.

\bigskip
{\noindent\bf Keywords:} protocol, HTTP, HTTP/2, Server Push

\vfill